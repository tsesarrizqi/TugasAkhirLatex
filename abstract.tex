%
% Halaman Abstract
%
% @author  Andreas Febrian
% @version 1.00
%

\chapter*{ABSTRACT}

\vspace*{0.2cm}

\noindent \begin{tabular}{l l p{11.0cm}}
	Name&: & \penulis \\
	Program&: & \programInggris \\
	Title&: & \judulInggris \\
\end{tabular} \\ 

\vspace*{0.5cm}

\noindent 
\textit{Deep Learning inference} saat ini sudah dapat dijalankan pada perangkat \textit{mobile}. Mayoritas \textit{library} untuk \textit{mobile Deep Learning} hanya memungkinkan operasi-operasi matriks pada \textit{inference} dijalankan menggunakan CPU. Penulis meneliti penggunaan GPU untuk menjalankan operasi-operasi matriks pada \textit{mobile Deep Learning inference}. Penulis telah mengimplementasikan beberapa \textit{kernel} tambahan pada Tensorflow Lite untuk operasi perkalian matriks-matriks dan konvolusi matriks yang berjalan di \textit{mobile} GPU ketika \textit{inference}. Penulis menggunakan OpenCL, API pemrograman paralel untuk berbagai jenis prosesor, untuk mengimplementasikan \textit{kernel} tersebut. Hasil pengujian yang dilakukan menunjukkan bahwa OpenCL \textit{kernels} yang berjalan di GPU untuk operasi perkalian matriks-matriks memiliki performa yang lebih baik daripada \textit{naive kernel} dan \textit{optimized kernel} dari Tensorflow Lite yang berjalan di CPU, terutama ketika matriks berukuran besar. Sementara itu, OpenCL \textit{kernel} untuk operasi konvolusi matriks mampu mengungguli performa \textit{naive kernel} dan \textit{optimized kernel} ketika banyaknya \textit{batch} dan kanal dari matriks \textit{output} cukup sedikit.
\\

\vspace*{0.2cm}

\noindent Keywords: \\ 
\noindent Deep Learning; Mobile GPU; OpenCL; \\

\newpage